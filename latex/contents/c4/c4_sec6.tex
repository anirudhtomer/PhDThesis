\section{Discussion}
\label{c4:sec:discussion}
In this paper, we presented a methodology to create personalized schedules for burdensome diagnostic \textit{tests} used to detect disease \textit{progression} in early-stage chronic non-communicable disease \textit{surveillance}. For this purpose, we utilized joint models for time-to-event and longitudinal data. Our approach first combines a patient's clinical data (e.g., longitudinal biomarkers) and previous invasive test results to estimate patient-specific cumulative-risk of disease progression over their current and future follow-up visits. We then plan future invasive tests whenever this cumulative-risk of progression is predicted to be above a certain threshold. We select the risk threshold automatically in a personalized manner, by optimizing a utility function of the patient-specific consequences of choosing a particular risk threshold based schedule. These consequences are, namely, the number of invasive tests (burden) planned in a schedule, and the expected time delay in detection of progression (shorter is beneficial) if the patient progresses. Last, we calculate this expected time delay in a personalized manner for both personalized and fixed schedules to assist patients/doctors in making a more informed decision of choosing a test schedule.

Using joint models gives us certain advantages. First, since joint models employ random-effects, the corresponding risk-based schedules are inherently personalized. Second, to predict this patient-specific risk of progression, joint models utilize all observed longitudinal measurements of a patient. Also, the continuous longitudinal outcomes are not discretized, which is commonly a case in Markov Decision Process and flowchart-based test schedules. Third, personalized schedules update automatically with more patient data over follow-up. Fourth, we calculated the expected number of tests (burden) and expected time delay in detecting progression (shorter is beneficial) in a patient-specific manner. Using our methodology, these can be calculated for both personalized and fixed schedules. Thus, patients/doctors can compare risk-based and fixed schedules and choose one according to their preferences for the expected burden-benefit ratio. Last, although this work concerns invasive test schedules in disease surveillance, the methodology is generic for use under a screening setting as well.

Personalized schedules that we proposed require a risk threshold. We optimized the threshold choice using a generic utility function based on the expected number of biopsies and time delay in detecting progression. We used only these two measures because they are easy to interpret but simultaneously critical for deciding the timing of invasive tests. Also, the time delay in detecting progression is an easily-quantifiable surrogate for the window of opportunity for curative treatment and additional benefits of observing progression early. Practitioners may extend/modify our utility function by adding to/replacing time delay with commonly used decision-theoretic measures such as quality-adjusted life-years/expectancy (QALY/QALE).

We evaluated personalized schedules in a full cohort via a realistic simulation of a randomized clinical trial for prostate cancer surveillance patients. We observed that personalized schedules reduced many unnecessary biopsies for non-progressing patients compared to the widely used annual schedule. This happened at the cost of simultaneously having a slightly longer time delay in detecting progression. Although, this delay should still be safe because it was almost equal to the delay of the world's largest prostate cancer active surveillance program PRIAS's schedule. The simulation study results are by no means the performance-limit of the personalized schedules. Instead, models with higher predictive accuracy and discrimination capacity than the PRIAS based model may lead to an even better balance between the number of tests and the time delay in detecting progression. As for the practical usability of the PRIAS based model in prostate cancer surveillance, despite the moderate predictive performance, we expect this model's overall impact to be positive. There are two reasons for this. First, the risk of adverse outcomes because of personalized schedules is quite low because of the low rate of metastases and prostate cancer specific mortality in prostate cancer patients~\citep{bokhorst2015compliance}. Second, studies~\citep{carvalho,inoue2018comparative} have suggested that after the confirmatory biopsy at year one of follow-up, biopsies may be done as infrequently as every two to three years, with limited adverse consequences. In other words, longer delays in detecting progression may be acceptable after the first negative biopsy.

There are certain limitations to this work. First, in practice, most cohorts have a limited study period. Hence, the cumulative-risk profiles of patients and resulting personalized schedules can only be created up to the maximum study period. For this problem, the risk prediction model should be updated with more follow-up data over time. The proposed joint model assumed all events other than progression to be non-informative censoring. Alternative models that account for competing risks may lead to better results as they estimate absolute and not the cause-specific risk of progression. The detection of progression is susceptible to inter-observer variation, e.g., pathologists may grade the same biopsy differently. Progression is sometimes obscured due to sampling error, e.g., biopsy results vary based on location and number of biopsy cores. Although models that account for inter-observer variation~\citep{balasubramanian2003estimation} and sampling error~\citep{coley2017prediction} will provide better risk estimates, the methodology for obtained personalized schedules can remain the same.