\section{Background}
\label{c7:sec:background}
Low-risk chronic non-communicable disease (e.g., localized prostate cancer, low-risk dysplasia) patients often undergo repeated invasive \emph{tests} (biopsies, endoscopies, etc.) for confirming disease \emph{progression}. A progression is a non-terminal event upon which patients usually undergo serious treatments, e.g., surgery, radiotherapy. Typically, invasive tests are conducted routinely according to a one-size-fits-all (e.g., yearly) fixed schedule~\citep{bokhorst2015compliance,choi2012screening,krist2007timing,mcwilliams2008surveillance,henderson2011surveillance}. Invasive tests are burdensome~\citep{loeb2013systematic,krist2007timing} but also indispensable for timely detection of disease progression. Specifically, frequent one-size-fits-all test schedules promise shorter time delays in observing progression at the cost of imposing an extra burden on patients who progress slowly. The vice versa holds for infrequent tests. Our aim in this thesis was to balance better the number of tests (burden) and time delay in detecting progression (shorter is beneficial) than fixed schedules. To this end, we developed and applied statistical methods for scheduling invasive diagnostic tests (e.g., biopsies, endoscopies) in a personalized manner.

To create personalized test schedules we first utilized a statistical model to predict a patient's cumulative-risk of progression over the whole follow-up period based on his accumulated clinical data. This risk profile manifested the transition of a patient's disease state over time from low-risk to progressed. Hence, subsequently, we used it to guide the timing of future invasive tests. Specifically, we derived personalized test schedules by optimizing utility functions of clinical parameters of interest (Chapters~\ref{c2}, \ref{c3}, and \ref{c4}) under the estimated patient-specific cumulative-risk of progression. We also employed the cumulative-risk of progression to assess the widely utilized approach of scheduling invasive tests using partially observable Markov decision processes (Chapter~\ref{c4:appendix:pomdp}). We then implemented personalized biopsy schedules for real patients of the seven largest prostate cancer active surveillance programs (Chapter~\ref{c5}) in a web-application (\url{https://emcbiostatistics.shinyapps.io/prias_biopsy_recommender/}). The use of personalized schedules is not limited to invasive tests only. In this regard, we also applied them for planning the NT-proBNP biomarker (a simple blood test) measurements in chronic heart failure patients (Chapter~\ref{c6}).