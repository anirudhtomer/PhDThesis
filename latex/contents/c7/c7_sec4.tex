\section{General Conclusion}
\label{c7:sec:conclusion}
In this work, we explored personalized schedules for invasive diagnostic tests in chronic non-communicable disease surveillance. To detect disease progression timely, in surveillance, typically invasive tests are planned in a one-size-fits-all manner, or flowcharts are used for test protocols. Neither of these methods exploit patient data fully. In contrast, the proposed personalized schedules rely on joint models for time-to-event and longitudinal data, which utilize complete patient data, including baseline covariates, longitudinal outcomes, and results of previous tests. The model building process is crucial in obtaining effective personalized schedules. Specifically, a model that predicts progression with a low error and a high capacity for discrimination will lead to personalized schedules that better balance the burden and benefit of repeated tests. 

Personalized schedules are not a panacea, and there is no single schedule that is suitable for all patients. In this regard, our methodology for estimating the expected number of tests and expected time delay in detecting progression in a patient-specific manner for any schedule can assist patients and doctors in shared decision making of an appropriate schedule. It is also essential to implement personalized schedules in Internet and web-applications, separately for each disease surveillance. Currently, we have done so for prostate cancer active surveillance patients. We hope this work will motivate surveillance studies to investigate personalized schedules more, e.g., via a randomized clinical trial.