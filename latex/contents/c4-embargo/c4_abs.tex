\begin{abstract}
Benchmark surveillance \textit{tests} for diagnosing disease \textit{progression} (e.g., biopsies, endoscopies) in early-stage chronic non-communicable diseases (e.g.,~cancer, lung diseases) are usually invasive. For detecting progression timely, patients undergo invasive tests planned in a fixed one-size-fits-all manner (e.g.,~annually). We present personalized test schedules based on progression-risk, that aim to optimize the number of tests (burden) and time delay in detecting progression (shorter is beneficial) better than fixed schedules. Our motivation comes from the problem of scheduling biopsies in prostate cancer surveillance.

Using joint models for time-to-event and longitudinal data, we consolidate patients' longitudinal data (e.g.,~biomarkers) and results of previous tests, into individualized future cumulative-risk of progression. We then create personalized schedules by planning tests on future visits where the predicted cumulative-risk is above a \textit{threshold} (e.g.,~5\% risk). We update personalized schedules with data gathered over follow-up. To find the optimal risk threshold, we minimize a utility function of the expected number of tests (burden) and expected time delay in detecting progression (shorter is beneficial) for different thresholds. We estimate these two in a patient-specific manner for following any schedule, by utilizing a patient's predicted risk profile. Patients/doctors can employ these quantities to compare personalized and fixed schedules objectively.
\end{abstract}