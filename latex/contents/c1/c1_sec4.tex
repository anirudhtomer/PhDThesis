\section{Outline of Thesis}
\label{c1:sec:thesis_outline}
The outline of the rest of this thesis is as follows. In Chapter~\ref{c2}, using loss functions from Bayesian decision theory, we develop a methodology for personalized biopsy decisions in prostate cancer active surveillance. In Chapter~\ref{c3}, we extend the joint model proposed in Chapter~\ref{c2} to account for both PSA and DRE longitudinal outcomes. Also, we focus exclusively on progression-risk based personalized biopsy decisions and conduct a more realistic simulation study than Chapter~\ref{c2}. In Chapter~\ref{c4}, we generalize our model for use surveillance across different chronic diseases and extend single optimal biopsy decisions to full optimal biopsy schedules. To this end, we define and utilize two measures of performance of a schedule. These are, namely, the expected number of invasive tests and the expected time delay in detecting progression. We evaluate the POMDP framework in Chapter~\ref{c4:appendix:pomdp}. We also apply our model and methodology in a real-world scenario. Specifically, in Chapter~\ref{c5}, we first externally validate a joint model fitted to the PRIAS prostate cancer dataset in six largest cohorts of the Movember Foundation's Global Action Plan Prostate Cancer Active Surveillance (GAP3) database. Then we implement the validated models and personalized schedules in a web-application. Lastly, in Chapter~\ref{c6}, we demonstrate the use of personalized schedules for planning biomarker measurements.