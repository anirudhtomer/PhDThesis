\section{Discussion}
\label{c6:sec:discussion}
This study aimed to optimally schedule NT-proBNP measurements for individual patients with CHF while maximizing the gain in prognostic information and reducing costs. Furthermore, to compare the efficacy of such personalized scheduling with fixed scheduling. We found that over a median follow-up time of 1.8 years, personalized scheduling required fewer NT-proBNP measurements per patient as compared to fixed scheduling while demonstrating similar performance regarding the prevention of adverse cardiac events. Since personalized scheduling required fewer measurements, this approach is expected to reduce related health care costs as well as patient burden compared to fixed scheduling.

The findings from our study carry important implications for future trials on biomarker-guided therapy. Previous biomarker-guided trials have generally used predefined sampling intervals and target levels~\citep{khan2018does,felker2017effect}. We show that, by using a personalized approach for scheduling NT-proBNP, timely intervention is enabled while using fewer NT-proBNP measurements as compared to a fixed schedule. Even though our fixed schedule consisted of rather frequent (trimonthly) NT-pro-BNP measurements, the high-risk interval identified by the personalized schedule was similar. On top of this, the fixed schedule was outperformed by the personalized schedule in terms of the number of measurements needed per patient to obtain this result. Maximizing information gain by estimating prognosis in an individual and optimal manner, while minimizing healthcare burden, may provide novel opportunities for timely adaptation of treatment. Future trials on natriuretic peptide-guided therapy for chronic heart failure may benefit from incorporating personalized screening intervals and personalized biomarker value targets; tailoring therapeutic interventions using this approach may reveal benefits that could not be demonstrated by previous trials, by nature of their design. 

Previous studies on personalized scheduling of blood sampling moments for measurements of biomarkers of disease are scarce, but this topic seems to be gaining attention recently. Personalized scheduling has been applied to patients undergoing aortic allograft root implantation~\citep{rizopoulos2016personalized}. Similarly to our study, this study used joint modeling. Aortic gradient levels were measured according to a fixed screening schedule. The authors demonstrated that personalized scheduling of aortic gradient assessments required fewer measurements and also performed better regarding the prevention of recurrent events as compared to fixed scheduling. Recently, personalized schedules for reducing the number of biopsies in low-risk prostate cancer patients have also been developed~\citep{tomer2019personalized}. Altogether, these promising results in other disease areas concur with our conclusion that personalized screening intervals carry the potential to improve patient monitoring and to ultimately individualize and herewith improve treatment.

\subsection{Limitations}
Several aspects of this study warrant consideration. First, we made several assumptions when developing the model, defining the thresholds, and setting up of the simulation study. However, in a sensitivity analysis, we performed the simulation study for three different risk thresholds, and the results remained essentially unchanged. Second, in our investigation, we performed a so-called demonstration, meaning that the analysis was performed on one `test' set of 50 patients, and was not repeated. We performed a demonstration because we aimed to provide a proof-of-concept here. A study with multiple test sets should be performed to validate our findings further. Although, it should be noted that such repeated estimations pose a heavy computational burden. Third, we did not account for the costs of implementation. Finally, while the concept of personalized screening intervals we present here seems promising, whether it would actually lead to the prevention of adverse events remains to be investigated in a clinical trial.

\subsection{Conclusions}
In conclusion, this study demonstrates for the first time that personalized scheduling of NT-proBNP measurements in patients with CHF, as compared to fixed scheduling, shows similar performance with regard to prevention of recurrent events but requires fewer NT-proBNP measurements. If such personalized scheduling were to be applied in natriuretic peptide-guided therapy, these benefits might translate into improved outcomes. Therefore, a clinical trial incorporating personalized scheduling should be considered.