\begin{abstract}
\textbf{Aims.} Personalized screening intervals for N-terminal pro-B-type natriuretic peptide (NT-proBNP) measurement in patients with chronic heart failure (CHF) could maximize information gain on individual patients' disease progression, while minimizing the number of necessary measurements. To improve prevention of clinical adverse events, we compared personalized scheduling of NT-proBNP measurements to fixed scheduling.

\textbf{Methods.} In 263 CHF patients from the Bio-SHiFT study, NT-proBNP was measured trimonthly according to a predefined fixed schedule. The primary endpoint (PE) comprised cardiac death, cardiac transplantation, left ventricular assist device implantation or heart failure hospitalization. We jointly modeled the repeated NT-proBNP measurements and PE. Using this fitted joint model, for each patient at each follow-up visit, we decided the optimal time point of the next NT-proBNP measurement based on the patient's individual NT-proBNP evolution. Personalized scheduling was compared to fixed scheduling by means of a simulation study, based on a replica of the Bio-SHiFT study population. Specifically, we compared the schedules' capability of identification of a high-risk interval (time-window with high risk preceding the PE; identification of its start enables appropriate timely intervention and prevention of PE occurrence), and number of measurements needed.

\textbf{Results.} Compared to fixed scheduling, personalized scheduling saved on average 2 measurements, while the start of the high-risk interval was similar by both approaches [personalized, Median:~6.6, IQR:~4.5-11.3; fixed, Median:~6.3, IQR:~4.2-10.3; months before occurrence of PE]. 

\textbf{Conclusion.} Personalized scheduling of NT-proBNP measurements in CHF patients, as compared to fixed scheduling, shows similar performance with regard to identification of impending adverse events, but requires fewer NT-proBNP measurements.
\end{abstract}