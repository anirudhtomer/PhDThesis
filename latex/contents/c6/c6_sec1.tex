\section{Introduction}
\label{c6:sec:introduction}
Circulating biochemical markers (biomarkers) may reflect the deterioration of patients with chronic heart failure (CHF) in an earlier stage than clinical assessment does. Hence, these biomarkers carry the potential to improve the risk stratification of patients with CHF and prevention of adverse clinical events~\citep{masson2008prognostic,gaggin2013biomarkers}. In the past decade, several trials on natriuretic peptide-guided therapy have been performed in which serial natriuretic peptide measurements were used to titrate medication~\citep{khan2018does,felker2017effect}. However, these trials have demonstrated inconclusive results. This may, in part, be explained by the fact that they mostly used predefined screening intervals (i.e., predefined time points) to assess biomarkers, as well as predefined target levels. Such predefined screening intervals and target levels do not account for individual temporal patterns of biomarkers, which may hamper their potential use for therapy guidance.

Conversely, a personalized screening approach that individualizes screening intervals and target levels based on individual temporal biomarker patterns may further improve risk assessment and therapy guidance. Such personalized screening intervals aim to maximize information gain on the individual patients' disease progression, while minimizing the necessary number of measurements, and therewith costs and patient burden~\citep{rizopoulos2016personalized}. In order to establish such intervals and targets, a model should be applied that incorporates detailed data on individual temporal patterns. Joint modeling is a statistical approach that takes into account full individual temporal patterns of biomarkers and links these patterns to the occurrence of adverse clinical events~\citep{rizopoulosJMbayes,rizopoulos2014tools}. In the Role of Biomarkers and Echocardiography in Prediction of Prognosis of Chronic Heart Failure Patients (Bio-SHiFT) study, we collected a median of 9 [interquartile range (IQR):~5--10] blood samples per patient. We demonstrated, by applying joint modeling, that individual temporal patterns of serially measured CHF-related biomarkers are associated with the prognosis of CHF patients~\citep{van2018toward}. Furthermore, we demonstrated that such a joint model when fitted on patients in Bio-SHiFT, could be used to estimate the patient-specific risk of the adverse outcome at each visit at the outpatient clinic. This risk is updated at each visit because it incorporates information on the patients' prognosis as derived from the newly available biomarker measurement~\citep{van2018toward}.

Subsequently, such a patient-specific risk profile, as derived from a joint model, can be applied to establish personalized screening intervals for future patients presenting at the outpatient clinic. This approach could contribute to improved prevention of further adverse clinical events. However, the benefits of this approach, over predefined screening intervals and targets, have not yet been investigated in CHF. Thus, in the current investigation, we aim to compare personalized scheduling to predefined fixed scheduling of N-terminal pro-B-type natriuretic peptide (NT-proBNP) measurements in individual CHF patients, in terms of the number of measurements performed according to each schedule, as well as the amount of time that remains for intervention before adverse outcome occurs. For this purpose, we use the data of the Bio-SHiFT study.