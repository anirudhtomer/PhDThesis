\section{Joint Model for Time-to-Event and Longitudinal Outcomes}
\label{c2:sec:jm_framework}
We start with a short introduction of the joint modeling framework we will use in our following developments. Let~$T_i^*$ denote the true GR time for the~$i$-th patient and let~$S$ be the schedule of his biopsies. Let the vector of the time of biopsies be denoted by~$T_i^S = \{T^S_{i0}, T^S_{i1}, \ldots, T^S_{i{N_i^S}}; T^S_{ij} < T^S_{ik}, \forall j<k\}$, where~$N_i^S$ are the total number of biopsies conducted. Because biopsy schedules are periodical,~$T_i^*$ cannot be observed directly and it is only known to fall in an interval~$l_i < T_i^* \leq r_i$, where~$l_i = T^S_{i{N_i^S - 1}}, r_i = T^S_{i{N_i^S}}$ if GR is observed, and~$l_i = T^S_{i{N_i^S}}, r_i=\infty$ if GR is not observed yet. Further let~$\boldsymbol{y}_i$ denote the~$n_i \times 1$ vector of PSA levels for the~$i$-th patient. For a sample of~$n$ patients the observed data is denoted by~$\mathcal{D}_n = \{l_i, r_i, \boldsymbol{y}_i; i = 1, \ldots, n\}$.

The longitudinal outcome of interest, namely PSA level, is continuous in nature and thus to model it the joint model utilizes a linear mixed effects model (LMM) of the form:
\begin{equation*}
\begin{split}
y_i(t) &= m_i(t) + \varepsilon_i(t)\\
&=\boldsymbol{x}_i^T(t) \boldsymbol{\beta} + \boldsymbol{z}_i^T(t) \boldsymbol{b}_i + \varepsilon_i(t),
\end{split}
\end{equation*}
where~$\boldsymbol{x}_i(t)$ and~$\boldsymbol{z}_i(t)$ denote the row vectors of the design matrix for fixed and random effects, respectively. The fixed and random effects are denoted by~$\boldsymbol{\beta}$ and~$\boldsymbol{b}_i$, respectively. The random effects are assumed to be normally distributed with mean zero and~$q \times q$ covariance matrix~$\boldsymbol{D}$. The true and unobserved, error free PSA level at time~$t$ is denoted by~$m_i(t)$. The error~$\varepsilon_i(t)$ is assumed to be t-distributed with three degrees of freedom and scale~$\sigma$, and is independent of the random effects~$\boldsymbol{b}_i$.

To model the effect of PSA on hazard of GR, joint models utilize a relative risk sub-model. The hazard of GR for patient~$i$ at any time point~$t$, denoted by~$h_i(t)$, depends on a function of subject specific linear predictor~$m_i(t)$ and/or the random effects:
\begin{align*}
h_i(t \mid \mathcal{M}_i(t), \boldsymbol{w}_i) &= \lim_{\Delta t \to 0} \frac{\mbox{Pr}\big\{t \leq T^*_i < t + \Delta t \mid T^*_i \geq t, \mathcal{M}_i(t), \boldsymbol{w}_i\big\}}{\Delta t}\\
&=h_0(t) \exp\big[\boldsymbol{\gamma}^T\boldsymbol{w}_i + f\{\mathcal{M}_i(t), \boldsymbol{b}_i, \boldsymbol{\alpha}\}\big], \quad t>0,
\end{align*}
where~$\mathcal{M}_i(t) = \{m_i(v), 0\leq v \leq t\}$ denotes the history of the underlying PSA levels up to time~$t$. The vector of baseline covariates is denoted by~$\boldsymbol{w}_i$, and~$\boldsymbol{\gamma}$ are the corresponding parameters. The function~$f(\cdot)$ parametrized by vector~$\boldsymbol{\alpha}$ specifies the functional form of PSA levels~\citep{brown2009assessing,rizopoulos2012joint,taylor2013real,rizopoulos2014bma} that is used in the linear predictor of the relative risk model. Some functional forms relevant to the problem at hand are the following: 
\begin{eqnarray*}
\left \{
\begin{array}{l}
f\{\mathcal{M}_i(t), \boldsymbol{b}_i, \boldsymbol{\alpha}\} = \alpha m_i(t),\\
f\{\mathcal{M}_i(t), \boldsymbol{b}_i, \boldsymbol{\alpha}\} = \alpha_1 m_i(t) + \alpha_2 m'_i(t),\quad \text{with}\  m'_i(t) = \frac{\mathrm{d}{m_i(t)}}{\mathrm{d}{t}}.\\
\end{array}
\right.
\end{eqnarray*}
These formulations of~$f(\cdot)$ postulate that the hazard of GR at time~$t$ may be associated with the underlying level~$m_i(t)$ of the PSA at~$t$, or with both the level and velocity~$m'_i(t)$ of the PSA at~$t$. Lastly,~$h_0(t)$ is the baseline hazard at time~$t$, and is modeled flexibly using P-splines. More specifically:
\begin{equation*}
\log{h_0(t)} = \gamma_{h_0,0} + \sum_{q=1}^Q \gamma_{h_0,q} B_q(t, \boldsymbol{v}),
\end{equation*}
where~$B_q(t, \boldsymbol{v})$ denotes the~$q$-th basis function of a B-spline with knots ${\boldsymbol{v} = v_1, \ldots, v_Q}$ and vector of spline coefficients~$\gamma_{h_0}$. To avoid choosing the number and position of knots in the spline, a relatively high number of knots (e.g., 15 to 20) are chosen and the corresponding B-spline regression coefficients~$\gamma_{h_0}$ are penalized using a differences penalty~\citep{eilers1996flexible}. Parameter estimation using the Bayesian approach is presented in Appendix~\ref{c2:appendix:A}.